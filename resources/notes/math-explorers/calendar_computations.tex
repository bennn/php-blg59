\documentclass{article}
\usepackage{fullpage, setspace, amsmath, minibox}
\begin{document}
  \title{Calendar Computations}
  \date{}
  \author{Ben Greenman (blg59)}
  \maketitle
  \onehalfspacing

  \renewcommand{\abstractname}{}%\large{Introduction}}
  \begin{abstract}
  \large
    Did you know that March 21, 2001 was a Wednesday? Or that January 1, 2076 will be a Friday?
    With a bit of math, one can determine the exact day of the week of any date: 
    past, present, or future. 
    Calculate the weekday on which the Declaration of Independence was signed (July 4, 1776), 
    man first set foot upon the moon (July 20, 1969), 
    or William Shakespeare was born (April 26, 1594). 
    Impress friends and family by naming the weekdays of important personal events.
  \end{abstract}

% Making whitespace
  \begin{enumerate} \item[] \item[] \item[] \end{enumerate}

  \tableofcontents
\newpage
  \section{Introduction to Modular Arithmetic}
    These calendar calculations are based on a concept called modular arithmetic. 
    In modular arithmetic, numbers do not go on forever. 
    Instead, they “reset” once they reach a certain value. 
    This value is called the \emph{modulus}. 
    Like the numbers on a clock or the number of days in a week, counting numbers 
    in modular arithmetic repeat a set pattern.

    For example, in a module 3 system, we would count 
    0, 1, 2, 0, 1, 2, 0, 1, 2, 0, 1, 2 . . . 
    and in a module 10 system, we would count 
    0, 1, 2, 3, 4, 5, 6, 7, 8, 9, 0, 1, 2, 3, . . . 
    Notice how the numbers start off like our counting numbers do, 
    beginning at zero and increasing one by one, but never reach the modulus. 
    In fact, the module 10 numbers match the numbers in the ones' place of our 
    base-10 counting numbers. 
    Compare our counting numbers to the module 10 system in the following table:\\\\
    \begin{center}
    \begin{tabular}{| c | c || c | c |}\hline
      Counting & Module 10 & Counting & Module 10 \\\hline\hline
      0 & 0 & 20 & 0 \\\hline
      1 & 1 & 21 & 1 \\\hline
      2 & 2 & 22 & 2 \\\hline
      3 & 3 & 23 & 3 \\\hline
      4 & 4 & 24 & 4 \\\hline
      5 & 5 & 25 & 5 \\\hline
      6 & 6 & 26 & 6 \\\hline
      7 & 7 & 27 & 7 \\\hline
      8 & 8 & 28 & 8 \\\hline
      9 & 9 & 29 & 9 \\\hline
      10 & 0 & 30 & 0 \\\hline
      11 & 1 & 31 & 1 \\\hline
      12 & 2 & 32 & 2 \\\hline
      13 & 3 & 33 & 3 \\\hline
      14 & 4 & 34 & 4 \\\hline
      15 & 5 & 35 & 5 \\\hline
      16 & 6 & 36 & 6 \\\hline
      17 & 7 & 37 & 7 \\\hline
      18 & 8 & 38 & 8 \\\hline
      19 & 9 & 39 & 9 \\\hline
    \end{tabular}
    \end{center}
\newpage
    In fact, we can define a new elementary operation 
    (like addition, subtraction, multiplication, and division) 
    called the modulus. We will use the symbol \% as shorthand for writing out 
    “modulus” or the shorter “mod”. The modulus of a number will be equal to that 
    number's value in a module system. For example, here are a few numbers modulus 10:\\

    \begin{center}
      \begin{tabular}{c c c}
        $39 \ \% \ 10 = 9$  & $\ \ \ $ & $11 \ \% \ 10 = 1$\\
        $5 \ \% \ 10 = 5$   & $\ \ \ $ & $400 \ \% \ 10 = 0$\\
        $405 \ \% \ 10 = 5$ & $\ \ \ $ & $444 \ \% \ 10 = 4$
      \end{tabular}
    \end{center}

    Notice how these equations match the table above.
    The number 39, in the module 10 system, has the value 9. 
    If you were to extend the table to reach 444, 
    you would find that the 444th value in the module 10 system is 4. 
    But there is a faster way to find any number's value in any module system.

    \begin{center}
      \textbf{For any two numbers A and B, A \% B is the remainder of A / B}
    \end{center}

    For the numbers A = 444 and B = 10, 444 \% 10 equals the remainder of $\frac{444}{10}$. 
    $\frac{444}{10} = 44$ remainder 4, so 444 \% 10 = 4. 
    In another example, 7 \% 5 = remainder of $\frac{7}{5}$. 
    $\frac{7}{5} = 1$ remainder 2, so 7 \% 5 = 2.\\
%    This is all we need to know to do the calendar computations, 
%    but if you want to learn more about modular arithmetic, 
%    check these articles at the Math Explorers' page (link) and Wikipedia (link).
%    Further reading section

    \begin{center}
    \fbox{
      \minibox{
        \textbf{Activity}: Compute the following modulus operations\\
        \small{Hint: You can use the google search bar to check your answers}\\\\
        \begin{tabular}{c c c c c c}
          $\ \ \ \ $ & 15 \% 5 & 1 \% 99 & 10 \% 3 & 25 \% 6 & 3 \% 1 \\
          $\ \ \ \ $ & 24 \% 7 & 32 \% 4 & 70 \% 5 & 14 \% 2 & 8 \% 5
        \end{tabular}
      }
    }
    \end{center}

\newpage
  \section{January $1^{st}$ of the $21^{st}$ century}
    Now that you know how to take the modulus of a number, 
    let's use that skill to figure what day of the week New Years Day was\textemdash
    or will be\textemdash for any year from 2000 to 2099. 
    This is a simple, quick calculation; with a little practice, 
    you'll be able to do it accurately without paper, pencil, or calculator! 
    More importantly, it's the first step toward finding the weekday of any date in history.
  
    First, we need to map the days of the week to numbers in a module system. 
    There are 7 days in a week, so we will use the module 7 system to build the mapping.\\

%    \begin{center}
%      \begin{tabular}{|c | c|}\hline
%        Sunday & 0 \\\hline
%        Monday & 1 \\\hline
%        Tuesday & 2 \\\hline
%        Wednesday & 3 \\\hline
%        Thursday & 4 \\\hline
%        Friday & 5 \\\hline
%        Saturday & 6 \\\hline
%      \end{tabular}
%    \end{center}
    \begin{center}
      \begin{tabular}{|c | c | c | c | c | c | c|}\hline
        Sunday & Monday & Tuesday & Wednesday & Thursday & Friday &  Saturday \\\hline
        0 & 1 & 2 & 3 & 4 & 5 & 6  \\\hline
      \end{tabular}
    \end{center}

    Use this table for reference as you learn these techniques, and memorize it. 
    If it helps, remember that Monday, with code 1, is the first (1) day of the 
    week that you have to go to school. Then the other days progress one-by-one 
    after Monday, and reset to zero (0) on Sunday.\\\\

    In order to find the day of the week of January $1^{st}$, $20ab$ 
    (where $a$ and $b$ are integers):
    \begin{itemize} \item[] \begin{enumerate}
      \item Take the two-digit number $ab$, add 1 and then divide the result by 4. 
            Ignore the remainder.\footnote{Example: $\frac{25 + 1}{4} = 6$ remainder 2. 
            Ignoring the remainder, our answer is 6.}
      \item Add the answer to the number $ab$
      \item Take the modulus 7 of this value
      \item Add 6 to the answer
      \item The day of the week is the day that matches this value in the table above
      \item[]
    \end{enumerate}\end{itemize}

    Here is a worked-out example for January 1st, 2067:
    \begin{itemize} \item[] \begin{enumerate}
      \item 2067 is of form 20ab, so 'ab' = 67.\\
            We add $1$ to 'ab' to get 68 \\
            $\frac{68}{4} = 17$ remainder 0. Let d = 17.
      \item $(ab + d) \ \% \ 7 = (67 + 17) \ \% \ 7$
            $67 + 16 = 84$, $84 \ \% \ 7 = 0$
      \item $0 + 6 = 6$
      \item 0 maps to Saturday. Jan 1, 2067 will be a Saturday
      \item[]
    \end{enumerate}\end{itemize}

    \begin{center}
      \fbox{
        \minibox{
          \textbf{Activity}: what day of the week was / will be New Year's Day in the years\\
          \begin{tabular}{c c c c c c c}
           $\ \ \ \ \ \ \ \ \ \ \ \ \ \ \ $ & 2001 & 2006 & 2020 & 2050 & 2052 & 2099
          \end{tabular}
        }
      }
    \end{center}

\newpage

  \section{Any weekday in the $21^{st}$ century}
  Now that we know how to find the day of Jan. $1^{st}$ in the $21^{st}$ century, 
  we can find a more general rule to work for any day of the $21^{st}$ century. 
  This process of starting from a specific case and working towards a general rule 
  is called \emph{induction}; it is a crucial tool for mathematicians.

  The computation here is a bit shorter than the one used for the special case above, 
  but requires two more data tables, in addition to the day codes.\\\\

  \begin{minipage}{0.5\linewidth}
    \centering
    \begin{tabular}{| c | c |}\hline
      Month & Code \\\hline\hline
      January & 6 (5 in leap years) \\\hline
      February & 2 (1 in leap years) \\\hline
      March & 2 \\\hline
      April & 5 \\\hline
      May & 0 \\\hline
      June & 3 \\\hline
      July & 5 \\\hline
      August & 1 \\\hline
      September & 4 \\\hline
      October & 6 \\\hline
      November & 2 \\\hline
      December & 4 \\\hline
    \end{tabular}
  \end{minipage}
  \begin{minipage}{0.5\linewidth}
    \centering
    \begin{tabular}{| c | c |}\hline
      Year & Code \\\hline\hline
      2000 & 0 \\\hline
      2001 & 1 \\\hline
      2002 & 2 \\\hline
      2003 & 3 \\\hline
      2004 & 5 \\\hline
      2005 & 6 \\\hline
      2006 & 0 \\\hline
      2007 & 1 \\\hline
      2008 & 3 \\\hline
      2009 & 4 \\\hline
      2010 & 5 \\\hline
      2011 & 6 \\\hline
    \end{tabular}
  \end{minipage}\\\\

  First, lets analyze the table of year codes. 
  There are only twelve codes in the table, but you can easily extend it. 
  Given the code of any year:
  \begin{itemize} \item[] \begin{enumerate}
    \item Add 1 to that code. Reset to zero if value equal 7.
    \item If the next year is a leap year, add 1 to the new code
          Just don't forget to “reset” the code when the value exceeds 6.
    \item[]
  \end{enumerate}\end{itemize}
  As an example, let's find the code for 2012. Given the code of 2011 = 6:
  \begin{itemize} \item[] \begin{enumerate}
    \item Add 1 to code of 2011. 6 + 1 = 7. Reset to 0.
    \item Is 2012 a leap year? $2012 \ \% \ 4 = 0$. It is a leap year. Add 1 to new code.
    \end{enumerate}
    \item[] The code for 2012 is 1.
  \end{itemize}
  Try working through the above steps on your own to find the code for 2013.

\newpage

  This is great, but what if we want the code for 2099? 
  Are we supposed to create a data table from 2000 to 2099? 
  Of course not! There's a faster way. 
  To determine the code for any year in the 21st century without creating a table:
  \begin{itemize} \item[] \begin{enumerate}
    \item Find the number of leap years before and including this year
    \item Add that result to the last two digits of the year
    \item Take the modulus 7 of the answer
    \item[]
  \end{enumerate}\end{itemize}
  For the year 2012, the last two digits are 12. 
  To find the number of leap years, we can work through a multiplication table:
  \begin{center}  
  \begin{tabular}{| c | c | c |}\hline
%    * & 1 & 2 & 3 \\\hline\hline
%    4 & 4 & 8 & 12 \\\hline
    $4 * 1 = 4$ & $4 * 2 = 8$ & $4 * 3 = 12$\\\hline
  \end{tabular}
  \end{center}

  Our last two digits = 12, so we stop here, but include that result. 
  Three (3) is the biggest number we found using the table, 
  so we conclude there have been 3 leap years so far in the 21st century. 
  Now we add the number of leap years and the last two digits of 2012 to get $3 + 12 = 15$.
  Final answer $= 15 \ \% \ 7 = 1$.\footnote{You could extend 
  the year table above to verify this answer.}

% Just making some whitespace
  \begin{itemize} \item[] \end{itemize}

  One more example: the year 2089. 
  First, we need to find the number of leap years up to and including 2089.
  We could work through the multiplication table as in the previous example, 
  or we could use division as a shortcut.
  $\frac{89}{4} = 22$ remainder 1. 
  4 goes in to 89 twenty-two times, so we conclude there were 22 leap years before 2089.
  Adding the last two digits to this answer, we get 89 + 22 = 111. 
  Final answer $= 111 \ \% \ 7 = 6$. The year code for 2089 is 6.

\newpage

  The month codes are more straightforward. 
  Except for January and February, whose values decrease by $1$ in leap years, 
  they do not change from year to year. 
  Nevertheless, memorizing the month codes presents a challenge. 
  The following mnemonics might help you remember:\footnote{But you're encouraged to create
  your own!}
  \begin{enumerate} \item[] \begin{itemize}
    \item January (6): January is the first full month of winter. 
          Winter has 6 letters \& January has code 6. 
    \item February (2): It's the 2nd month, and Groundhog's Day is Feb. 2nd. 
          February has code 2.
    \item March (2): In like a lion, out like a lamb. 
          One lion + one lamb = 2 animals. March has code 2. 
    \item April (5): Five letters in “April,” April has code 5.
    \item May (0): Cinco de Mayo is in May. Think “May-O”. May has code 0.
    \item June (3): June is the first of 3 months of summer vacation. June has code 3
    \item July (5): On Independence Day, fireworks fly. Think “fireworks fly five.” 
          July has code 5. 
    \item August (1): August begins with an “A”, the 1st letter of the alphabet. 
          A is \#1, August has code 1. 
    \item September (4): Fall begins in September. Fall has 4 letters \& September has code 4.
    \item October (6): 666 is the number of the beast. 
          Demons play on Halloween, October has code 6 
    \item November (2): Election day is always a Tuesday in November. Tuesday, 2's-day. 
          November, 2. 
    \item December (4): December is the holiday month. 
          Holidays for all. 4-all. December is code 4
    \item[]
  \end{itemize} \end{enumerate}

  Putting it all together, you can get the weekday of a date by taking the modulus 7 of 
  the sum of a date's: year code + month code + day. 
  The month code and year code come from the tables above; the day is simply the day. 
  To find which day of the week August 12, 2009 fell on, just solve:
  
  \begin{center}
    (Year Code + Month Code + Day Code) \% 7 \\
    month code = 1\\
    day = 12\\
    year code = 4\\
    ( 1 + 12 + 4 ) \% 7\\
    17 \% 7 = 3\\
    3 maps to Wednesday. August 12, 2009 was a Wednesday.\\
  \end{center}

% whitespace
  \begin{enumerate} \item[] \item[] \end{enumerate}

  \begin{center}
    \fbox{
      \minibox{
        \textbf{Activity}: Find day of the week of these dates\\
        \begin{tabular}{c c c c c}
          Mar 12, 2031 & Feb. 9, 2006 & Nov. 2, 2099 & July 4, 2042 & May 28, 2053
        \end{tabular}
      }
    }
  \end{center}

\newpage

  \section{Compute any weekday in any century}
  Here it is, the main attraction: determine the day of the week of 
  any date in the history of the world! 
  We'll take our knowledge from the previous sections and add one last generalization.

  So far we have tables for day codes, month codes, and 21st century year codes. 
  We need a new table that will give us the year codes for the first year in any century. 
  Let's call these “century codes” for short.
  Recall that we found the year codes for the 21st century by 
  finding the day code of January $1^{st}$, 2000 and advancing that value by 1. 
  We could apply this same rule to find the year codes for the first year of any century, 
  but we would first need to know the day of the week that January 1st fell on each century. 
  You can either look online and research these dates, or you can use the following formula:

  \begin{center}
  \textbf{For any century, let the first two digits equal C. 
  Then the century code for\\ 
  January 1st of that century is: (1 + (2 * (3 – (C \% 4)))) \% 7.}\\
  \end{center}

  Working step-by-step:
  \begin{itemize} \item[] \begin{enumerate}
    \item Take the first two digits of the century to get C (For 1983, C = 19)
    \item Solve C \% 4
    \item Subtract: 3 minus answer from step 2
    \item Multiply: 2 * your answer from step 3
    \item Add 1 to this value
    \item Take modulus 7 for final result
    \item[]
  \end{enumerate}\end{itemize}

  Using the 21st century (2000 – 2099):
  \begin{itemize} \item[] \begin{enumerate}
    \item $C = 20$
    \item $C \ \% \ 4= 0$
    \item $3–0= 3$
    \item $2*3= 6$
    \item $6+ 1= 7$
    \item $7 \ \% 7= 0$
    \item[]
  \end{enumerate}\end{itemize}

  In conclusion, the century code for the 21st century is 0. 
  Notice that this number matches the year code for 2000. 
  Also, the 6 above (result of $2 * 3 – (C\%4)$) is the day code of January 1st, 2000. 
  This is no coincidence!

\newpage

  One more example: let's find the code for the 18th century (1700-1799)
  \begin{itemize} \item[] \begin{enumerate}
    \item $C = 17$
    \item $C \ \% \ 4= 1$
    \item $3–1= 2$
    \item $2*2= 4$
    \item $4 + 1 = 5$
    \item $5 \ \% \ 7 = 5$
  \end{enumerate}\end{itemize}
  The century code for the 18th century is 5.

  \begin{center}
    \fbox{
      \minibox{
        \textbf{Activity}: 
        Computer the century codes for the $19^{th}$ and $20^{th}$ centuries
      }
    }
  \end{center}

  \begin{center}
  \begin{tabular}{| c | c |}\hline
    Century & Code \\\hline\hline
    $18^{th} (1700-1799)$ & 5 \\\hline
    $19^{th} (1800-1899)$ &   \\\hline
    $20^{th} (1900-1999)$ &   \\\hline
    $21^{st} (2000-2099)$ & 0 \\\hline
  \end{tabular}
  \end{center}

\newpage

    We're almost done! The last step is to find the year code for any year. 
    Recall how we found the year codes for any year in the 21st century without using a table.
    \begin{itemize} \item[] \begin{enumerate}
      \item Take the last 2 digits of the year
      \item Add the number of leap years up to \& including that year
      \item Take the modulus 7 of the answer
      \item[]
    \end{enumerate} \end{itemize}
    Using these century codes, the formula to get the day of the week of any day in history is:
    \begin{center}
      (Century Code + Year Code + Month Code + Day) \% 7
    \end{center}

    One last example problem: find the day of the week of July 4, 1776.
    \begin{center}
      Century Code = 5\\
      Year Code = (76 + ($\frac{76}{4}$)) \% 7 = (76 + 19) \% 7 = 95 \% 7 = 4\footnote{I 
        took a shortcut here. See the term $\frac{76}{4}$? 
        That's to find the number of leap years up to and including 1776. 
        Normally, we want to ignore the remainder of this answer, but 4 divides evenly into 76,
        so there is no remainder. If the term were 77 / 4 = 19 remainder 1, 
        we would drop the remainder and arrive at the same answer of 19.}\\
      Month Code = 5\\
      Day = 4\\
    Final Answer = (Century code + Year code + Month code + Day) \% 7 (5 + 4 + 5 + 4) \% 7\\
    18 \% 7\\
    4\\
    \end{center}
    The day code 4 matches Thursday. 
    Congress approved the Declaration of Independence on a Thursday!

    With practice, you'll memorize the century codes, month codes, and day codes. 
    Then when someone asks you the weekday of a date (from the past or future), 
    all you'll need to do is calculate the year code and put it all together.

    \begin{center}
      \fbox{
        \minibox{
          \textbf{Activity}: Find the weekday these dates fell on\\
            \begin{tabular}{c c c}
              Oct 27, 1858 & May 5, 1862 & April 9, 1865\\
              Dec. 15 1911 & April 14, 1912 & Dec 7, 1941\\
            \end{tabular}
          Bonus: Research the significance of these dates
        }
      }
    \end{center}

\newpage

  \section{Why Does This Work?}
  We are mathematicians. We want to understand why things work. 
  “Just because” is not an acceptable answer, and we don't believe in magic. 
  We seek explanations—even if they contradict our sense of what is “right” or “normal”. 
  So what's the rule behind these calendar computations?
  \begin{enumerate}
    \item Each week has 7 days. Each year has 365 or 366 days. 
          365 and 366 are very close to 364, which is a multiple of 7. 
          So any day that takes place on a Monday one year will take place on a Tuesday 
          if the next year is not a leap year, and on a Wednesday if it is a leap year. 
          Notice that these “steps”, +1 for normal years and +2 for leap years, 
          match the “steps” we used in the table of year codes.
          As a side note, if our years had 364 days 
          (or any other number of days X such that X \% 7 = 0), 
          every date would fall on the same day of the week every year!
    \item Modular arithmetic makes the calculations possible for any date 
          (not just numbers less than 7). 
          It's a fascinating topic, but beyond the scope of this article. You are 
          strongly encouraged to study it further!
  \end{enumerate}
\end{document}
