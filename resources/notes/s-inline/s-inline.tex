\documentclass{article}

\usepackage{hyperref}
\usepackage{palatino}
\usepackage{amsmath}
\usepackage{amsthm}
\usepackage{setspace}

\newtheorem*{theorem}{Theorem}

\newcommand{\summary}[1]{
\renewcommand{\maketitle}{{%
\flushleft\textsf{Ben Greenman \hfill \today} \\
\textsf{\hfill #1} \\
\vspace{1mm}
\hrule
\vspace{0.4cm}}}
\maketitle
}

\newcommand{\swtable}[2]{
\vspace{0.5cm}
\subsubsection*{Strengths}
\begin{itemize}
#1
\end{itemize}
\vspace{0.1cm}

\vspace{0.5cm}
\subsubsection*{Weaknesses}
\begin{itemize}
#2
\end{itemize}
\vspace{0.1cm}
}

\newcommand{\questions}[1]{
\vspace{0.5cm}
\subsubsection*{Questions}
\begin{itemize}
#1
\end{itemize}
\vspace{0.1cm}
}
\newcommand{\comments}[1]{
\vspace{0.5cm}
\subsubsection*{Comments}
\begin{itemize}
#1
\end{itemize}
\vspace{0.1cm}
}



\begin{document}
\summary{Inline Expansion: \emph{when} and \emph{how}?}

Goal: Faster code without too much size increase~\cite{s-inline}.

\subsubsection*{when}
The proposed inlining metric compares the code size of a function call
 against the code size of to-be-inlined function body and additionally
 measures the expected gain upcoming optimizations might achieve if the
 inlining did happen.

All these decisions happen statically.
The compiler first picks a total size parameter $k$, to prevent huge/recursive
 expansion, and then analyzes functions in random order.


\subsubsection*{how}
Non recursive functions: just let-bind the parameters and paste the inlined function below.

Recursive: introduce a label for the recursive call \& inline into a loop.

Lastly, the paper remarks on a loop unrolling heuristic.
Add a new label then unroll as long as the inlining fuel $k$ allows.


On a few small programs, the algorithms appear to trade time for space evenly
 (with slight advantage to time\textemdash you get faster faster than you get bigger).


\questions{
\item
  Why are function call sequences so expensive in these languages?
  Can we make them faster\textemdash more like {\tt goto} statements for tail calls?
}

\footnotesize
\bibliographystyle{plain}
\bibliography{s-inline}
\end{document}
