\documentclass{article}

\usepackage{hyperref}
\usepackage{listings}
\usepackage{palatino}
\usepackage{amsmath}
\usepackage{amsthm}
\usepackage{mathpartir}
\usepackage{graphicx}
\usepackage{setspace}
\usepackage{stmaryrd}

\newtheorem*{theorem}{Theorem}
\newtheorem*{challenge}{Challenge}
\newtheorem*{defn}{Definition}

\renewcommand{\maketitle}{{%
\flushleft\textsf{Ben Greenman}
\textsf{\hfill 2016-02-29 \hfill Outline: Interpolants in Model Checking} \\
\vspace{1mm}
\hrule
\vspace{0.4cm}}}

\newcommand{\fvs}[1]{\emph{atoms}(#1)}

\lstset{
  basicstyle=\ttfamily,
  breaklines=true,
  language=C,
  numbers=left
}

\newcommand{\todo}[1]{\textbf{TODO: #1}}


\begin{document}
\summary{On the Criteria To Be Used in Decomposing Systems into Modules}

The paper compares two methods of breaking a program into components:
 the first is to divide the program into steps (or rather, a flowchart)
 and the second is modularize related behaviors~\cite{p-criteria}.
These strategies roughly correspond to imperative and functional programming.\footnote{There are hints of object-oriented programming, but the core idea is just of a module with a clear API of functions.}
The argument is that the second method\textemdash of encapsulating design decisions and presenting abstract interfaces\textemdash is better from an engineering standpoint and sufficently motivating to inspire work on efficient compilation.


\swtable{
\item
  Identifies and clearly states an important problem: how should we organize large software projects?
\item
  The \emph{Changeability} section raised interesting points.
  Design decisions 2, 4, and 5 are still relevant (time/space tradeoffs).
}{
\item
  The worked example is nice, but I would have liked more anecdotes from larger projects.
\item
  The paper does not deeply consider modularizing across a language, libraries, and user applications.
  (Looking back, it's obvious that the programming language should provide abstractions for strings, character sets, and I/O.
   User applications should \emph{never} provide these things.
   At the very least, they belong in a library outside the ``system''.)
}

It's hard to imagine taking over a week to write a KWIC index.
Also, why is ``KWIC index'' a keyword for the paper?

\footnotesize
\bibliographystyle{plain}
\bibliography{p-criteria}
\end{document}
