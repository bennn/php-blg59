\documentclass{article}

\usepackage{hyperref}
\usepackage{listings}
\usepackage{palatino}
\usepackage{amsmath}
\usepackage{amsthm}
\usepackage{mathpartir}
\usepackage{graphicx}
\usepackage{setspace}
\usepackage{stmaryrd}

\newtheorem*{theorem}{Theorem}
\newtheorem*{challenge}{Challenge}
\newtheorem*{defn}{Definition}

\renewcommand{\maketitle}{{%
\flushleft\textsf{Ben Greenman}
\textsf{\hfill 2016-02-29 \hfill Outline: Interpolants in Model Checking} \\
\vspace{1mm}
\hrule
\vspace{0.4cm}}}

\newcommand{\fvs}[1]{\emph{atoms}(#1)}

\lstset{
  basicstyle=\ttfamily,
  breaklines=true,
  language=C,
  numbers=left
}

\newcommand{\todo}[1]{\textbf{TODO: #1}}


\begin{document}
\summary{Gradual Typing for Functional Languages}

This paper introduces the catchphrase ``gradual typing'' and describes a cast-based type system that admits untyped terms~\cite{st-gradual}.
Type safety in this system means that if a term may be assigned a type (with ``dynamic'' holes), then it will evaluate to either a value or a run-time cast error.
To handle diverging terms, the authors parameterize evaluation by a natural number limiting the number of computation steps.

\qquad The paper is written ``by th' book''~\cite{p-tapl} and reads nicely.
Moreover, all proofs are formalized in Isabelle.


\swtable{
\item Excellent title.
\item Clear presentation.
\item Casts are a neat solution to mutable references and untyped functions.
  (Rather than striving to preserve an invariant, delay the check until necessary.)
}{
\item
  The benefits of the untyped language are not well motivated (besides terms like ``flexible'' etc).
  In particular, I would like to know when gradual typing might be more useful than type inference.
\item
  Eliding blame is a mistake.
  A $\mathsf{CastError}$ is no better than a segfault if it does not accurately tell where the problem occurred.
}


\footnotesize
\bibliographystyle{plain}
\bibliography{st-gradual}
\end{document}
