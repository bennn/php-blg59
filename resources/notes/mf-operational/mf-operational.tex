\documentclass{article}

\usepackage{hyperref}
\usepackage{palatino}
\usepackage{amsmath}
\usepackage{amsthm}
\usepackage{setspace}

\newtheorem*{theorem}{Theorem}

\newcommand{\summary}[1]{
\renewcommand{\maketitle}{{%
\flushleft\textsf{Ben Greenman \hfill \today} \\
\textsf{\hfill #1} \\
\vspace{1mm}
\hrule
\vspace{0.4cm}}}
\maketitle
}

\newcommand{\swtable}[2]{
\vspace{0.5cm}
\subsubsection*{Strengths}
\begin{itemize}
#1
\end{itemize}
\vspace{0.1cm}

\vspace{0.5cm}
\subsubsection*{Weaknesses}
\begin{itemize}
#2
\end{itemize}
\vspace{0.1cm}
}

\newcommand{\questions}[1]{
\vspace{0.5cm}
\subsubsection*{Questions}
\begin{itemize}
#1
\end{itemize}
\vspace{0.1cm}
}
\newcommand{\comments}[1]{
\vspace{0.5cm}
\subsubsection*{Comments}
\begin{itemize}
#1
\end{itemize}
\vspace{0.1cm}
}



\begin{document}
\summary{Operational Semantics for Multi-Language Programs}

This paper gives semantics for language interoperability~\cite{mf-operational}.\footnote{The title is a complete description.}
The presentation is divided into three sections: initially foreign values are considered opaque, next they are given meaning in a type system, and finally they are monitored by higher-order contracts.
The benefits and drawbacks of each approach are clearly stated.


\swtable{
\item
  Identifies the problem of defining \emph{type-preserving} semantics between higher-order languages.
\item
  Offers three reasonable solutions
  and makes connections to pragmatic
  industry solutions. Even the ``na\"ive'' lump embedding had practical
  applications before this paper gave it a firm semantics.
\item
  Bold use of color. This might be the paper's most enduring academic contribution.
}{
\item
The Scheme and ML variants are very similar. Although the authors
consider the problem of different integer encodings, real languages vary
in many ways. For example:
\begin{itemize}
\item
 User-defined datatypes in one language may not have an obvious
 encoding in another language. Can the (non-lump) semantics be
 extended with user-defined encodings for these data structures,
 and include a check that the encodings are isomorphisms?
\item
 Coq and Agda assume that all functions are pure and depend
 heavily on this assumption. Languages like OCaml, however, are
 stateful. This makes linking OCaml and Coq code potentially
 dangerous. Can we extend the semantics with similar higher-level
 properties?
\item
 Potentially non-terminating FFI calls can also violate one
 language's invariants. For example, the LiquidHaskell language
 uses an SMT solver to solve type-level equations, and must take
 care to not send diverging variables to the constraint solver. It does
 so by statically analyzing Haskell code. Could this paper's
 semantics offer termination guarantees for FFI calls?
\item
 {\tt C} and {\tt C++} are plainly unsafe; they can read and write to arbitrary
 memory locations. Is there hope for a semantics that rules out
 memory model violations? At the very least, can we detect such
 abuses? I'm imagining a C program that corrupts ML's
 contract-checking code.
\end{itemize}
}

\footnotesize
\bibliographystyle{plain}
\bibliography{mf-operational}
\end{document}
