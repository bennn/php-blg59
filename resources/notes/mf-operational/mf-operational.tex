\documentclass{article}

\usepackage{hyperref}
\usepackage{palatino}
\usepackage{amsmath}
\usepackage{amsthm}
\usepackage{setspace}

\newtheorem*{theorem}{Theorem}

\newcommand{\summary}[1]{
\renewcommand{\maketitle}{{%
\flushleft\textsf{Ben Greenman \hfill \today} \\
\textsf{\hfill #1} \\
\vspace{1mm}
\hrule
\vspace{0.4cm}}}
\maketitle
}

\newcommand{\swtable}[2]{
\vspace{0.5cm}
\subsubsection*{Strengths}
\begin{itemize}
#1
\end{itemize}
\vspace{0.1cm}

\vspace{0.5cm}
\subsubsection*{Weaknesses}
\begin{itemize}
#2
\end{itemize}
\vspace{0.1cm}
}

\newcommand{\questions}[1]{
\vspace{0.5cm}
\subsubsection*{Questions}
\begin{itemize}
#1
\end{itemize}
\vspace{0.1cm}
}
\newcommand{\comments}[1]{
\vspace{0.5cm}
\subsubsection*{Comments}
\begin{itemize}
#1
\end{itemize}
\vspace{0.1cm}
}



\begin{document}
\summary{Operational Semantics for Multi-Language Programs}

This paper gives semantics for language interoperability~\cite{mf-operational}.\footnote{The title is a complete description.}
The presentation is divided into three sections: initially foreign values are considered opaque, next they are given meaning in a type system, and finally they are monitored by higher-order contracts.
The benefits and drawbacks of each approach are clearly stated.


\swtable{
\item
  Identifies the problem of defining \emph{type-preserving} semantics between higher-order languages.
\item
  Bold use of color.
\item
  Good motivation, includes many references to industry projects for language interop.
}{
\item
  The ML and Scheme languages are very similar.
  How would this scale if ML added references or Scheme had \texttt{call/cc}?
\item
  Similarly, how about transitive interop?
  Or even fixpoints?
  But it makes sense to leave a full exploration of those design spaces to future work.
}

\footnotesize
\bibliographystyle{plain}
\bibliography{mf-operational}
\end{document}
