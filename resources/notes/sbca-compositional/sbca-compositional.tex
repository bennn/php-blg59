\documentclass{article}

\usepackage{hyperref}
\usepackage{listings}
\usepackage{palatino}
\usepackage{amsmath}
\usepackage{amsthm}
\usepackage{mathpartir}
\usepackage{graphicx}
\usepackage{setspace}
\usepackage{stmaryrd}

\newtheorem*{theorem}{Theorem}
\newtheorem*{challenge}{Challenge}
\newtheorem*{defn}{Definition}

\renewcommand{\maketitle}{{%
\flushleft\textsf{Ben Greenman}
\textsf{\hfill 2016-02-29 \hfill Outline: Interpolants in Model Checking} \\
\vspace{1mm}
\hrule
\vspace{0.4cm}}}

\newcommand{\fvs}[1]{\emph{atoms}(#1)}

\lstset{
  basicstyle=\ttfamily,
  breaklines=true,
  language=C,
  numbers=left
}

\newcommand{\todo}[1]{\textbf{TODO: #1}}


\begin{document}
\summary{Compositional CompCert}

Defines \emph{interaction semantics} and uses them to prove that a 13-pass compiler
 is correct and link-safe.

Correctness means that the target program shows the same trace of behaviors as the source.
Observable behaviors are halting, external calls, memory writes, and memory frees. %\footnote{Reads aren't ok. Writes&Frees are effects, and newly-proved!}[

Linking is defined as a function taking a set of interaction semantics to one large interaction semantics (with a heterogenous call stack).
Link safety means that (A) linking a whole program, then compiling it and (B) compiling a set of programs, then linking them give the same result.

The interaction semantics are the crucial part of this work, as both correctness \& link-safety are stated in terms of them.
An interaction semantics is a 5-state operational model of a program.
The states are: {\tt initial}, {\tt running}, {\tt halted}, {\tt at\_external}, and {\tt after\_external}.


\swtable{
\item
  They did it! Go Princeton!
\item
  I think it's a strength that there's nothing really complicated in the proofs \& proof strategies.
  They did the straightfoward, meticulous thing; in my opinion justifying the ``informal'' reasoning compiler writers would previously do.
}{
\item
  What problems does this solve?
  Section 4 lists two requirements for external calls: do not rely on memory that will be removed by the compiler or memory introduced by the compiler.
  Is this year-long, thousand-line effort just for that?
  If so, what did we learn to make future proofs easier (besides to keep a very detailed memory model).
\item
  They removed optimizations!
  The optimizations are the main selling point of CompCert over other (non-Coq) high-assurance compilers.
  These compositional techniques \emph{need} to scale to the full CompCert compiler.
\item
  (Not a flaw of the paper, but) Section 6 says ``we have not yet applied much proof automation at all, so we believe there is room for improvement''~\cite{sbca-compositional}.
  The original CompCert paper made a similar claim.
  Still, I have yet to see \emph{any} large Coq project where proof were within 2x lines of code.
  Where are the proof automation success stories?
}

\footnotesize
\bibliographystyle{plain}
\bibliography{sbca-compositional}
\end{document}
