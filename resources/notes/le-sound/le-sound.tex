\documentclass{article}

\usepackage{hyperref}
\usepackage{palatino}
\usepackage{amsmath}
\usepackage{amsthm}
\usepackage{setspace}

\newtheorem*{theorem}{Theorem}

\newcommand{\summary}[1]{
\renewcommand{\maketitle}{{%
\flushleft\textsf{Ben Greenman \hfill \today} \\
\textsf{\hfill #1} \\
\vspace{1mm}
\hrule
\vspace{0.4cm}}}
\maketitle
}

\newcommand{\swtable}[2]{
\vspace{0.5cm}
\subsubsection*{Strengths}
\begin{itemize}
#1
\end{itemize}
\vspace{0.1cm}

\vspace{0.5cm}
\subsubsection*{Weaknesses}
\begin{itemize}
#2
\end{itemize}
\vspace{0.1cm}
}

\newcommand{\questions}[1]{
\vspace{0.5cm}
\subsubsection*{Questions}
\begin{itemize}
#1
\end{itemize}
\vspace{0.1cm}
}
\newcommand{\comments}[1]{
\vspace{0.5cm}
\subsubsection*{Comments}
\begin{itemize}
#1
\end{itemize}
\vspace{0.1cm}
}



\begin{document}
\summary{Sound Type-Dependent Syntactic Language Extension}

Presents a framework for defining typed languages and extending them with syntactic sugar~\cite{le-sound}.
The sugarings can extend the valid expressions and types of the language.

All syntax extensions are guaranteed type-sound, and are typechecked before they elaborate into core syntax.
Thus any type errors are caught early and presented to the user in terms of the surface syntax.

\swtable{
\item
  Promotes (some form of) extensible type systems.
\item
  Preserves extensions' abstraction barriers, probably makes errors easy to debug.
}{
\item
  Language extensions must be expressible in the core; growth is limited by the language you begin with.
\item
  The system is no help to existing languages.
  Before a language can grow, it must be defined within the authors' framework.
}

Hmph, this makes me think macrotypes would've had a good shot at POPL.

\footnotesize
\bibliographystyle{plain}
\bibliography{le-sound}
\end{document}
