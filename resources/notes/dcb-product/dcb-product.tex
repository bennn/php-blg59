\documentclass{article}

\usepackage{hyperref}
\usepackage{palatino}
\usepackage{amsmath}
\usepackage{amsthm}
\usepackage{setspace}

\newtheorem*{theorem}{Theorem}

\newcommand{\summary}[1]{
\renewcommand{\maketitle}{{%
\flushleft\textsf{Ben Greenman \hfill \today} \\
\textsf{\hfill #1} \\
\vspace{1mm}
\hrule
\vspace{0.4cm}}}
\maketitle
}

\newcommand{\swtable}[2]{
\vspace{0.5cm}
\subsubsection*{Strengths}
\begin{itemize}
#1
\end{itemize}
\vspace{0.1cm}

\vspace{0.5cm}
\subsubsection*{Weaknesses}
\begin{itemize}
#2
\end{itemize}
\vspace{0.1cm}
}

\newcommand{\questions}[1]{
\vspace{0.5cm}
\subsubsection*{Questions}
\begin{itemize}
#1
\end{itemize}
\vspace{0.1cm}
}
\newcommand{\comments}[1]{
\vspace{0.5cm}
\subsubsection*{Comments}
\begin{itemize}
#1
\end{itemize}
\vspace{0.1cm}
}



\begin{document}
\summary{Product Lines of Theorems}

Gives a framework for composable verification~\cite{dcb-product}.
Proof scripts follow the same modular structure as the actual code, and compose in a similar manner.

How do the proofs compose?
Each component documents its assumptions in an interface.
New components must satisfy this interface to work together.
The approach is implemented in Coq by modeling the Java language.

\swtable{
\item
  Suggests a discipline, not a particular framework or library.
\item
  Old proofs do not need to be re-checked as new features are added.
}{
\item
  Proof assumptions need to find a sweet spot between generality and domain-specifics
   because new features need to fit within these assumptions.
  (Unless you go back and refactor.)
\item
  Need to work in Coq, and build your codebase from the ground up.
}

\footnotesize
\bibliographystyle{plain}
\bibliography{dcb-product}
\end{document}
