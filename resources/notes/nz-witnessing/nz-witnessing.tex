\documentclass{article}

\usepackage{hyperref}
\usepackage{listings}
\usepackage{palatino}
\usepackage{amsmath}
\usepackage{amsthm}
\usepackage{mathpartir}
\usepackage{graphicx}
\usepackage{setspace}
\usepackage{stmaryrd}

\newtheorem*{theorem}{Theorem}
\newtheorem*{challenge}{Challenge}
\newtheorem*{defn}{Definition}

\renewcommand{\maketitle}{{%
\flushleft\textsf{Ben Greenman}
\textsf{\hfill 2016-02-29 \hfill Outline: Interpolants in Model Checking} \\
\vspace{1mm}
\hrule
\vspace{0.4cm}}}

\newcommand{\fvs}[1]{\emph{atoms}(#1)}

\lstset{
  basicstyle=\ttfamily,
  breaklines=true,
  language=C,
  numbers=left
}

\newcommand{\todo}[1]{\textbf{TODO: #1}}


\begin{document}
\summary{Witnessing Program Transformations}

Proposes \emph{witness generation} procedures for proving program transformations~\cite{nz-witnessing}.
The main insight is that heuristic passes can produce evidence to help the verifier.

From high-level to low-level, the paper's connection to real life is:
\begin{itemize}
\item Compiler optimizations are program transformations.
\item A program transformation is correct if running the result (target) gives an output
 that the source code could have produced.
\item
  Formally, target code $T$ \emph{implements} source code $S$ according to a relation $\alpha$
   iff $\alpha$ relates all initial target states with initial source states
   and whenever the target reduces to a final state, the source can reduce to a final state related by $\alpha$.
  Note that this definition ignores intermediate steps.
  An optimizing pass that produced $T$ given $S$ is correct if we can give a relation $\alpha$ with the above property.
\item
  The paper argues that \emph{stuttering simulations} are a useful class of relations
   because they are straightforward to prove and imply \emph{implements} relations.
  If $S$ and $T$ are in a stuttering simulation, then $T$ implements $S$.
\item
  A stuttering simulation $R$ must:
  \begin{itemize}
  \item Relate initial target states to initial source states.
  \item
    Provide a well-founded measure $<$ such that, for all transitions $t_1 \rightarrow t_2$ in $T$
     and all source states $s_1$ related by $R$ to $t_1$, one of the following must hold:
    \begin{itemize}
    \item There is a source state $s_2$ such that $s_1 \rightarrow s_2$ and $(t_2, s_2) \in R$
    \item We stutter in $T$, which means $(t_2, s_1) \in R$ and $(t_2, s_1) < (t_1, s_1)$ holds.
    \item We stutter in $S$; meaning is a $s_2$ such that $(t_2, s_2) \in R$ and $(t_2, s_2) < (t_1, s_2)$.
    \end{itemize}
    The stuttering means we don't move to a new state along both axis, but instead decrease along $<$.
  \end{itemize}
\item
  Stuttering similations compose, just like passes compose into a whole compiler.
\item
  Stuttering similation can prove interesting optimizations.
  The authors show examples of constant propogation, dead code elimination, CFG compression, loop hoisting, and loop reordering.
\end{itemize}

\swtable{
\item
  It's exciting to hope that we could validate all compiler passes in a uniform way.
  It's also exciting that we don't need to rewrite old optimizing transformations to do so (we only need to modify them to produce evidence).
}{
\item
  I'm not certain how the wellfounded relations $<$ compose.
  Do we get to reset the measure between passes?
\item
  The examples are explained well, but I'm looking forward to a more general way of producing these relations during an optimization pass.
}

\footnotesize
\bibliographystyle{plain}
\bibliography{nz-witnessing}
\end{document}
