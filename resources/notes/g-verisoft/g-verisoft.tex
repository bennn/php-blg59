\documentclass{article}
\usepackage[margin=1in]{geometry}
\usepackage{hyperref}
\usepackage{palatino}
\usepackage{amsmath}
\usepackage{amsthm}
\usepackage{setspace}

\newtheorem*{theorem}{Theorem}

\newcommand{\summary}[1]{
\renewcommand{\maketitle}{{%
\flushleft\textsf{Ben Greenman \hfill \today} \\
\textsf{\hfill #1} \\
\vspace{1mm}
\hrule
\vspace{0.4cm}}}
\maketitle
}

\newcommand{\swtable}[2]{
\vspace{0.5cm}
\subsubsection*{Strengths}
\begin{itemize}
#1
\end{itemize}
\vspace{0.1cm}

\vspace{0.5cm}
\subsubsection*{Weaknesses}
\begin{itemize}
#2
\end{itemize}
\vspace{0.1cm}
}

\newcommand{\questions}[1]{
\vspace{0.5cm}
\subsubsection*{Questions}
\begin{itemize}
#1
\end{itemize}
\vspace{0.1cm}
}
\newcommand{\comments}[1]{
\vspace{0.5cm}
\subsubsection*{Comments}
\begin{itemize}
#1
\end{itemize}
\vspace{0.1cm}
}



\begin{document}
\summary{Model Checking for Programming Languages with VeriSoft}
\onehalfspacing

The paper gives a high-level description of stateless search and advertises
 a model checker called VeriSoft that implements stateless search of C
 programs~\cite{g-verisoft}.
VeriSoft is capable of quickly exploring large state spaces.
This makes it possible to directly check C programs for deadlocks and
 assertion violations rather than models of C programs.

Stateless search is a relaxation of classical state-space exploration.
Instead of recording all visited states during the search (eventually caching
 the entire state space in memory), stateless search records only the current
 path and a few additional states\textemdash no more than will fit in memory.
The search is then guided by the recorded states, ideally to
 minimize the number of transitions needed to find bad states.

The main difficulty is choosing which states to remember and leveraging this
 information to avoid exploring redundant paths.
To this end, the paper gives an algorithm that records \emph{sleep sets}
 and computes \emph{persistent sets} to yield an efficient search.
At a high level, these sets recognize independent sequences of transitions and
 then exclude all possible interleavings of independent (i.e. commuting) transitions.
However these key techniques are not explained in detail.
Sleep sets are defined in~\cite{g-using} and~\cite{gp-refining} shows how to compute
 persistent sets.

Overall the author's stateless search seems promising.
The translation from program to model (or model to runnable program) always
 makes me nervous, though I understand the need of abstracting the pragmatic
 details of real languages.
So I am glad that VeriSoft's model is the program itself.
But I would have liked to see more details and evidence in this paper; in
 particular an algorithm for computing sleep sets \& persistent sets, and
 more evaluation than the dining philosophers and one bug report.

\questions{
\item
  The model of a concurrent system presented in Section 2 is deterministic.
  Isn't this too restrictive?
  For one, it would be useful to check parallel programs.
  But more likely, if I have a bug in my supposedly-deterministic C program
   that introduces non-determinism, I would like VeriSoft to find it\textemdash
   would it?
\item
  I do not understand the statement that non-determinism is ``not found in
   programming languages''.
  C definitely had random number generation in 1997 and it has always been prone
   to non-deterministic memory errors: suppose I have a program that
   allocates an array and reads off the end of it\textemdash what is the result?
\item
  Is VeriSoft still in use today? I search Google quickly, but only found
   academic papers.
}

%% \comments{
%% \item
%%   Yes! The program is the model!
%% \item
%%   Visible \& invisble ops remind me of observations and actions
%% \item
%%   wtf VS\_toss. No random-number generators before this?
%% }

\footnotesize
\bibliographystyle{plain}
\bibliography{g-verisoft}
\end{document}
