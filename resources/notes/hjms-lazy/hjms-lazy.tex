\documentclass{article}

\usepackage{hyperref}
\usepackage{listings}
\usepackage{palatino}
\usepackage{amsmath}
\usepackage{amsthm}
\usepackage{mathpartir}
\usepackage{graphicx}
\usepackage{setspace}
\usepackage{stmaryrd}

\newtheorem*{theorem}{Theorem}
\newtheorem*{challenge}{Challenge}
\newtheorem*{defn}{Definition}

\renewcommand{\maketitle}{{%
\flushleft\textsf{Ben Greenman}
\textsf{\hfill 2016-02-29 \hfill Outline: Interpolants in Model Checking} \\
\vspace{1mm}
\hrule
\vspace{0.4cm}}}

\newcommand{\fvs}[1]{\emph{atoms}(#1)}

\lstset{
  basicstyle=\ttfamily,
  breaklines=true,
  language=C,
  numbers=left
}

\newcommand{\todo}[1]{\textbf{TODO: #1}}


\begin{document}
\summary{Lazy Abstraction}
\onehalfspacing

Instead of trying to prove a program property by building a model, checking if the property holds, refining the model and repeating,
 the paper suggests proving the property in ``one'' lazy pass by incrementally building a model~\cite{hjms-lazy}.
Initially the model is not very specific, but gets refined lazily as the property is checked based on where the proof fails to succeed.
The point of this one-pass, incremental approach is that it saves re-doing work by running the 3-step loop.

Their main algorithm is very general.
Given (essentially) a region structure $R$ (representing sets of program states),
 5 well-typed operations regarding $R$ ($\sqcup$, $\sqcap$, \emph{pre}, \emph{post}, $\llbracket\cdot\rrbracket$),
 and a \emph{focus} operation for refining an abstract region to exclude specific impossible paths,
 the algorithm explores the region and either returns an error trace or a proof of satisfiability.
Interestingly, there are no laws or extra properties that $R$ and the 5 operations
 must satisfy, though the algorithm is not guaranteed to terminate unless unequal
 program states can be distinguished in finitely many steps and the $\sqcup$
 operation is well-founded.

The algorithm is implemented for C programs in the \textsc{Blast} model checker.
The authors briefly explain how \textsc{Blast} is implemented and show
 high-level results of using \textsc{Blast} to validate small device driver programs.
One interesting detail is that the tool's runtime is dominated by calls to a
 theorem prover (Simplify) and not the lazy abstraction algorithm.
That is very good.


\questions{
\item
  Why couldn't they handle the C multiplication operator?
  Is that just a limitation of the theorem prover they worked with?
\item
  Evaluating the device driver code used fewer predicates than I was
   expecting.
  I thought we might have 3-5 predicates per branch in the program
   rather than 3-5 total.
  And so, I wonder what these predicates look like.
\item
  What is a CFA? It looks like a control flow graph, but the anagram is wrong.
}

\comments{
\item
  The level of abstraction used in Sections 3 and 4 didn't help me understand the paper;
   I and would have preferred a more concrete / restricted exposition.
  On one hand it's good that the algorithm can do more than
   verify C programs, but it wasn't clear to me what those other application
   areas might be.
  Without being able to imagine other applications,
   the presentation seemed like abstraction for abstraction's own sake.
}

\footnotesize
\bibliographystyle{plain}
\bibliography{hjms-lazy}
\end{document}
