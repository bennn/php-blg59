\documentclass{article}

\usepackage{hyperref}
\usepackage{listings}
\usepackage{palatino}
\usepackage{amsmath}
\usepackage{amsthm}
\usepackage{mathpartir}
\usepackage{graphicx}
\usepackage{setspace}
\usepackage{stmaryrd}

\newtheorem*{theorem}{Theorem}
\newtheorem*{challenge}{Challenge}
\newtheorem*{defn}{Definition}

\renewcommand{\maketitle}{{%
\flushleft\textsf{Ben Greenman}
\textsf{\hfill 2016-02-29 \hfill Outline: Interpolants in Model Checking} \\
\vspace{1mm}
\hrule
\vspace{0.4cm}}}

\newcommand{\fvs}[1]{\emph{atoms}(#1)}

\lstset{
  basicstyle=\ttfamily,
  breaklines=true,
  language=C,
  numbers=left
}

\newcommand{\todo}[1]{\textbf{TODO: #1}}


\begin{document}
\summary{A Critique of the Remote Procedure Call Paradigm}


\begin{quote}
  ``Meaningful distinctions must be preserved.'' --E. Bishop
\end{quote}

The RPC model aims to make local and remote procedure calls indistinguishable.
Tanenbaum and van Renesse argue that hiding the distinction between remote and local calls has many drawbacks~\cite{tv-euteco-1988}.

In particular, RPCs \emph{significantly differ} from local calls because:

\begin{itemize}
\item RPCs can \emph{fail} due to connection errors;
\item RPCs have higher \emph{latency};
\item RPCs run on a \emph{different} machine.
\end{itemize}

The possible \emph{failures} mean programmers must guard remote calls with error-handling code.
It is not clear, however, what the programmer should do to recover from a general RPC failure.
If the client/server connection fails after the server performs a stateful update, the client should not necessarily repeat the call.
If the server dies, clients may need to re-initialize state on the server.
If the client dies, the server may want to kill the orphaned process.

The \emph{latency} makes RPCs unsuitable in real-time applications or within high-frequency loops.
An RPC in the former case may add semantics-altering delays; the authors give device driver timeouts as an example.
An RPC in the latter case may add large performance overhead.

The \emph{remote} aspect of an RPC makes serialization and memory errors possible.
Programmers cannot assume the server has access to the same pointers and global variables as the client machine.
Nor can programmers assume floating point numbers have the same precision on the server, or that structures have the same alignment.

The authors furthermore note that RPCs are blocking, cannot make broadcast notifications, and cannot yield results in a stream.
In summary the RPC paradigm is useful in some cases, but is unsuitable for general-purpose distributed computing.


\footnotesize
\bibliographystyle{plain}
\bibliography{tv-euteco-1988}
\end{document}
