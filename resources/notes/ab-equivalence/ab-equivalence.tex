\documentclass{article}

\usepackage{hyperref}
\usepackage{palatino}
\usepackage{amsmath}
\usepackage{amsthm}
\usepackage{setspace}

\newtheorem*{theorem}{Theorem}

\newcommand{\summary}[1]{
\renewcommand{\maketitle}{{%
\flushleft\textsf{Ben Greenman \hfill \today} \\
\textsf{\hfill #1} \\
\vspace{1mm}
\hrule
\vspace{0.4cm}}}
\maketitle
}

\newcommand{\swtable}[2]{
\vspace{0.5cm}
\subsubsection*{Strengths}
\begin{itemize}
#1
\end{itemize}
\vspace{0.1cm}

\vspace{0.5cm}
\subsubsection*{Weaknesses}
\begin{itemize}
#2
\end{itemize}
\vspace{0.1cm}
}

\newcommand{\questions}[1]{
\vspace{0.5cm}
\subsubsection*{Questions}
\begin{itemize}
#1
\end{itemize}
\vspace{0.1cm}
}
\newcommand{\comments}[1]{
\vspace{0.5cm}
\subsubsection*{Comments}
\begin{itemize}
#1
\end{itemize}
\vspace{0.1cm}
}



\begin{document}
\summary{An Equivalence-Preserving CPS Translation via Multi-Language Semantics}

Defines a CPS translation from a simply-typed language to a language with polymorphism that preserves simply-typed contextual equivalence~\cite{ab-equivalence}.
If two source terms are contextually equal, then no target context will be able to distinguish translated versions of the terms.

\swtable{
\item
  Introduces / showcases many interesting techniques: using target-language types to encode source-language guarantees,
  using polymorphism for information hiding, back translation via partial evaluation.
\item
  Relates two languages of varying power.
}{
\item
  Is this going to scale beyond $\lambda$-calculi?
  The conclusion notes potential trouble with non-termination and recursive types; effects are another issue.
  Also, the fact that partial evaluation in the target yields a basically-source term seems more of a quirk than a useful observation.
\item
  The introduction gave a motivating background story about the pitfalls of language-based security, but then the paper only talked about purely functional languages.
  I wasn't convinced that security is a major issue for pure languages.
\item
  I don't understand the criticisms towards game semantics and domain theory in the introduction and conclusion.
  The first priority ought to be finding \emph{any} technique that scales all the way to a real language\textemdash say GHC or SML/NJ.
  In celebration we can worry about finding beautiful proofs and intuitive explanations, but until then there's no sense in discouraging other ideas.
}

The original definition of \emph{ciu} equivalence required contexts to use their hole term at least once~\cite{mt-equivalence,p-howe}.

\footnotesize
\bibliographystyle{plain}
\bibliography{ab-equivalence}
\end{document}
