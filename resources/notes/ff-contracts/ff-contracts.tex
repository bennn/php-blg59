\documentclass{article}

\input{'../def'}

\begin{document}
\summary{Software Contracts for Higher-Order Functions}

This paper introduces higher-order contracts for software development~\cite{ff-contracts}.
Its contributions include:
\begin{itemize}
\item A typed calculus modeling higher-order contracts \& their interactions with a type system
\item A protocol for assigning blame (when a contract is broken)
\item Proofs of contract soundness and compiler correctness
\item Notes on a full implementation of the calculus
\end{itemize}

\qquad Ordinary contracts are predicates used to assert the pre- and post-conditions of a function.
For example, a function representing the sequence of prime numbers can assert that it takes natural-number arguments and returns a prime number.
These predicates, however, are limited to first-order assertions.
Higher-order contracts support assertions about higher-order functions.
Now a curried division function can assert that when given a real-valued argument, it returns a function on the non-zero real numbers.

\qquad Moreover, the paper addresses the issue of \emph{blame} for higher-order contracts.
For first-order contracts, it is simple to tell which party is responsible when a contract assertion fails.
Either the function caller or the function itself is to blame, and we can stop execution immediately.
Higher-order contracts are not checked immediately, therefore we must have a protocol for re-constructing the chain of function calls that led to a contract violation.
The paper gives one such a protocol, and reports on the authors' experience using this protocol to build an IDE.

\qquad In closing, the authors recommend contracts at the coarsest level possible: module boundaries.
This should minimize the number of calls to contract-protected functions, encourage modular development, and enforce API documentation.

\swtable{
\item Clear motivation via examples \& practical experience.
\item The model demonstrates how this work would apply to languages besides Scheme.
}{
%% \item Blame should be an identifier, not a string
\item No quantitative evaluation of the Dr.Scheme implementation (how many contracts were dependent, flat, etc ...)
\item No discussion of future work! I would have liked to compare the authors' predictions with the +10-year research effort that really followed.
}

\subsubsection*{Notes}
%% Most influential paper for 2002
%% http://www.sigplan.org/Awards/ICFP/

It's funny to me that at least 4 times in this paper, and at least 1 time in his ICFP'14 keynote, Robby emphasizes that contracts are meant to help guide type systems research.
I still think the common opinion is that types \& contracts are opponents.

\qquad At least in Racket, the contracts mostly seem to enforce simple types\footnote{According to Michael Greenberg's thesis~\cite{g-manifest}, 82\% of Racket standard library contracts expressed simple types in 2011.} and until Brian LaChance arrived this summer, you could not even write contracts in Typed Racket.

\footnotesize
\bibliographystyle{plain}
\bibliography{ff-contracts}

\end{document}

