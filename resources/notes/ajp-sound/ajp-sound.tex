\documentclass{article}

\usepackage{hyperref}
\usepackage{palatino}
\usepackage{amsmath}
\usepackage{amsthm}
\usepackage{setspace}

\newtheorem*{theorem}{Theorem}

\newcommand{\summary}[1]{
\renewcommand{\maketitle}{{%
\flushleft\textsf{Ben Greenman \hfill \today} \\
\textsf{\hfill #1} \\
\vspace{1mm}
\hrule
\vspace{0.4cm}}}
\maketitle
}

\newcommand{\swtable}[2]{
\vspace{0.5cm}
\subsubsection*{Strengths}
\begin{itemize}
#1
\end{itemize}
\vspace{0.1cm}

\vspace{0.5cm}
\subsubsection*{Weaknesses}
\begin{itemize}
#2
\end{itemize}
\vspace{0.1cm}
}

\newcommand{\questions}[1]{
\vspace{0.5cm}
\subsubsection*{Questions}
\begin{itemize}
#1
\end{itemize}
\vspace{0.1cm}
}
\newcommand{\comments}[1]{
\vspace{0.5cm}
\subsubsection*{Comments}
\begin{itemize}
#1
\end{itemize}
\vspace{0.1cm}
}



\begin{document}
\summary{Sound Modular Verification of C Code Executing in an Unverified Context}

The authors dream of combining a fully-abstract compiler with protected-memory hardware to enable modular verification of C programs via separation logic.
We readers are dazzled with a proof that static Hoare logic assertions are never invalidated \emph{at runtime} and impressive benchmark results ($<\!\!10\%$ overhead) from a closed-source OCaml binary.
In all, a compelling work of fan fiction~\cite{ajp-sound}.

\swtable{
\item
  Clever combination of features.
  I hope that hardware designers soon give us a fast way to check memory integrity.
\item
  Very well-written introduction.
}{
\item
  The references to (modular) verification tools were good motivation;
   I would have also liked citations to companies using these tools.
  Especially, I would have liked to know of companies writing high-security code
   that still use ``untrusted'' code\textemdash either as a library or because they cannot afford complete verification.
\item
  The macro-benchmarks were not very convincing; {\tt gcc -O3} is far from fully-abstract!
  More detailed Sancus results would have been more useful than the benchmarks in \S 7.2.
\item
  On that note, I don't think ``fast'' and ``fully-abstract'' will ever coexist.
  The latter rules out too many of the optimizations.
  Shame on these guys for trying to keep us programming in C~{\tt ;)}.
\item
  In \S 7.4, I don't understand how copying the memory region could be cheaper than hashing.
  Is copying less expensive than multiplication? Maybe they need a faster (rolling) hash algorithm.
\item
  Blame is \emph{not} orthogonal to this work.
  Part of their informal guarantee is that bugs triggered in unverified code are detected.
  This idea of correctly blaming the unverified code should be formalized.
}

Who is P. W. O\&\#39;Hearn?

\footnotesize
\bibliographystyle{plain}
\bibliography{ajp-sound}
\end{document}
