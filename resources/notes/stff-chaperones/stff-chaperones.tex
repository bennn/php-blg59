\documentclass{article}

\usepackage{hyperref}
\usepackage{listings}
\usepackage{palatino}
\usepackage{amsmath}
\usepackage{amsthm}
\usepackage{mathpartir}
\usepackage{graphicx}
\usepackage{setspace}
\usepackage{stmaryrd}

\newtheorem*{theorem}{Theorem}
\newtheorem*{challenge}{Challenge}
\newtheorem*{defn}{Definition}

\renewcommand{\maketitle}{{%
\flushleft\textsf{Ben Greenman}
\textsf{\hfill 2016-02-29 \hfill Outline: Interpolants in Model Checking} \\
\vspace{1mm}
\hrule
\vspace{0.4cm}}}

\newcommand{\fvs}[1]{\emph{atoms}(#1)}

\lstset{
  basicstyle=\ttfamily,
  breaklines=true,
  language=C,
  numbers=left
}

\newcommand{\todo}[1]{\textbf{TODO: #1}}


\begin{document}
\summary{Chaperones and Impersonators: Run-time Support for Reasonable Interposition}

A \emph{chaperone} is a proxy limited to stateful behavior.
For instance a chaperone can record a log of all accesses to a vector, but it may not change the value returned by an access.

An \emph{impersonator} is an unrestricted proxy limited to mutable data structures.
Impersonators can do anything with their inputs and give any output.

These are really great, and they have low overhead~\cite{stff-chaperones}.
Possibly because the core implementation is in the C runtime system.

\footnotesize
\bibliographystyle{plain}
\bibliography{stff-chaperones}
\end{document}
