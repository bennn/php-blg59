\documentclass{article}

\usepackage{hyperref}
\usepackage{palatino}
\usepackage{amsmath}
\usepackage{amsthm}
\usepackage{setspace}

\newtheorem*{theorem}{Theorem}

\newcommand{\summary}[1]{
\renewcommand{\maketitle}{{%
\flushleft\textsf{Ben Greenman \hfill \today} \\
\textsf{\hfill #1} \\
\vspace{1mm}
\hrule
\vspace{0.4cm}}}
\maketitle
}

\newcommand{\swtable}[2]{
\vspace{0.5cm}
\subsubsection*{Strengths}
\begin{itemize}
#1
\end{itemize}
\vspace{0.1cm}

\vspace{0.5cm}
\subsubsection*{Weaknesses}
\begin{itemize}
#2
\end{itemize}
\vspace{0.1cm}
}

\newcommand{\questions}[1]{
\vspace{0.5cm}
\subsubsection*{Questions}
\begin{itemize}
#1
\end{itemize}
\vspace{0.1cm}
}
\newcommand{\comments}[1]{
\vspace{0.5cm}
\subsubsection*{Comments}
\begin{itemize}
#1
\end{itemize}
\vspace{0.1cm}
}



\begin{document}
\summary{Chaperones and Impersonators: Run-time Support for Reasonable Interposition}

A \emph{chaperone} is a proxy limited to stateful behavior.
For instance a chaperone can record a log of all accesses to a vector, but it may not change the value returned by an access.

An \emph{impersonator} is an unrestricted proxy limited to mutable data structures.
Impersonators can do anything with their inputs and give any output.

These are really great, and they have low overhead~\cite{stff-chaperones}.
Possibly because the core implementation is in the C runtime system.

\footnotesize
\bibliographystyle{plain}
\bibliography{stff-chaperones}
\end{document}
