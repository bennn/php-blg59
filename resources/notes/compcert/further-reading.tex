\section{Learn More}

%% \subsection{Conference Proceedings}
%% A timeline:

%% \begin{tabular}{l l}
%%   2006 & backend, frontend
%% \\     & coinductive opsem
%% \\     & optimizations based on dataflow
%% \\2007 & list-machine benchmark
%% \\     & zenon prover
%% \\     & cps transformations
%% \\     & separation logic for cminor
%% \\2008 & translation validators
%% \\2009 & faster coalescing
%% \\     & lazy code motion
%% \\2010 & regalloc and splitting
%% \\     & pipelining validator
%% \\2011 & flight controllers
%% \\2012 & alias analysis
%% \\     & lr1 parsers
%% \\     & optimized flight controllers
%% \\2013 & floating points (make section)
%% \end{tabular}

%% \subsection{Coinductive Opsem}
%% bigstep preferred, but cannot do nontermination~\cite{l-coinductive}.
%% Coinductive definitions and theorems.

%% \subsection{Dataflow Analysis}
%% %% http://gallium.inria.fr/~xleroy/publi/proofs_dataflow_optimizations.pdf
%% Optimizations as instances of dataflow problems~\cite{bgl-structured}.
%% General functor, uses Kildall's algorithm.

%% %% TODO where in source?
%% \begin{lstlisting}
%% (* F       = Function declarations
%%  * Map.A B = Map from keys of type A to values of type B
%%  *)
%% Module Type ANALYSIS.
%%   Parameter D : Set.
%%   (* The domain of the analysis *)
%%   Parameter A_f : F -> Map.T D.
%%   (* Analysis function; computes domain properties of
%%    * program points in the function F. *)
%%   Parameter A_P : D -> S -> Prop.
%%   (* Compatibility of concrete states in S with domain elements *)
%%   Parameter A_{entry} : \forall f S, A_P(A_F(f_{ep}), S).
%%   (* All states should be compatible with their abstract entry point *)
%%   Parameter A_{correct} : \forall f c1 c2 S1 S2,
%%     c1 \in f_c => (S1, f_C(c1)) \mapsto (S2, c2) => A_P(A_F(c1), S1) => A_P(A_F(c2), S2)
%%   (* All computation steps c1->c2 should map to compatible states
%% \end{lstlisting}

%% Kildall : initially compute D for each program point.
%%   Update successors with their bound joined with the current node.
%%   Reach fixedpoint because graph is finite and lattice is well-founded

%% (best to describe the 3 optimzations, then talk about Kildall for how-to-prove)


%% \subsection{List Machine}
%% %% http://gallium.inria.fr/~xleroy/publi/listmachine-lfmtp.pdf
%% Benchmark, appel + leroy
%% A second poplmark.
%% Not immediately interesting.


%% \subsection{Zenon Prover}
%% Based on tableau method
%% first-order logic with equality
%% for Focal environment, doing OCaml for execution and Coq for certification
%% seems to be going too slowly


%% \subsection{Separation Logic for Cminor}
%% %% http://www.ensiie.fr/~blazy/AppelBlazy07.pdf
%% Modifications to Cminor.
%% Just read the source.


%% \subsection{Formal Verification of Translation Validators}
%% %% http://gallium.inria.fr/~xleroy/publi/validation-scheduling.pdf
%% mach optimizations? re-check when doing passes


%% \subsection{Verified Validation of Lazy Code Motion}
%% %% http://gallium.inria.fr/~xleroy/publi/validation-LCM.pdf
%% Apparently not part of public CompCert.
%% Oh wait, entails CSE and hoisting


%% \subsection{Flight control}
%% %% https://hal.archives-ouvertes.fr/inria-00551370/document
%% %% https://hal.inria.fr/hal-00653367/document
%% Kind of interesting, standards requirements flowcharts

%% \subsection{Floating Point}
%% %% https://hal.inria.fr/hal-00743090v2/document


\subsection{OPLSS Tutorial}

In \href{http://gallium.inria.fr/~xleroy/courses/Eugene-2011/}{2011}
 and \href{http://gallium.inria.fr/~xleroy/courses/Eugene-2012/}{2012},
 Xavier Leroy taught a short course on verified compilers at the Oregon Programming Languages Summer School.\footnote{\url{https://www.cs.uoregon.edu/research/summerschool/}}
The course presents a small compiler for the IMP programming language and proves a few optimizations.\footnote{As formalized in the Software Foundations textbook.~\cite{pcgghsy-software}}
To summarize:
\begin{itemize}
\item
  The source language has assignment statements, if statements, and while loops.
\item
  The target language is a JVM-inspired stack language.
\item
  Basic compiler correctness follows from two theorems.
  \begin{itemize}
  \item If the source program terminates in state $\sigma$, then the compiled code will also terminate in $\sigma$.\footnote{A state is just a map from identifiers to variables. Source and target code use the same representation for states.}
    
  \item If the source program diverges, then so does the compiled code.
  \end{itemize}
\item
  Dead-code elimination ({\tt dce}) is implemented by approximating the live variables in a program and removing all assignments to dead variables.
  (This is a source-to-source translation.)
  A correct {\tt dce} pass ensures that optimized programs produce ``the same'' result state as un-optimized programs, where ``the same'' means two states hold the same values for all live variables.
  \begin{lstlisting}[style=Coq]
  Theorem dce_correct_terminating:
    forall st c st', c / st || st' ->
    (* Program `c` started in state `st` evaulates to state `st'` *)
    forall L st1,               (* `L` is a set of live variables *)
    agree (live c L) st st1 ->  (* `live` computes a fixpoint of `L` *)
    exists st1', dce c L / st1 || st1' /\ agree L st' st1'.
  \end{lstlisting}
  For diverging programs, the correctness theorem includes a well-founded measure on commands and uses small-step semantics.
  Then an instruction either progresses to a new state or deducts from the measure.
\item
  Register allocation is validated; the compiler checks that mapping on variables preserves semantics, taking special cases into account.
  (An external program should compute the allocation, but this is not part of the material.)
  To make register allocation and dead-code elimination compose, the statement of {\tt agree} is generalized over a valid map between variables.
\item
  Two static analyzers are implemented\textemdash one simple, the other with widening and inverse analysis of arithmetic and boolean operations.
  These analyzers are proven sound, and then used to implement constant propogation.
\end{itemize}
