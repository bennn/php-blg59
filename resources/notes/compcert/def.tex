\usepackage{palatino}
\usepackage{listings}
\usepackage{xcolor}
\usepackage{bbold}
\usepackage{graphicx}
\usepackage{tikz}
\usepackage{multicol}
\usepackage[margin=1in]{geometry}
\usepackage[colorlinks=true,citecolor=blue]{hyperref}

\renewcommand{\maketitle}{{\flushleft\large\textsf{Ben Greenman \hfill CompCert Overview \\ 2015-11-19 \\\vspace{1mm}\hrule}}}

\lstset{
  %% backgroundcolor=\color{white},
  alsoletter={&},
  breakatwhitespace=false,
  breaklines=true,
  captionpos=b,
  commentstyle=\color{red},
  escapeinside={\%*}{*)},
  %frame=single,
  keywordstyle=\color{blue},
  basicstyle=\ttfamily,
  %% identifierstyle=\color{green},
  language=C,
  %% numbers=left,
  %% numbersep=5pt,
  numberstyle=\color{black},
  showspaces=false,
  showstringspaces=false,
  showtabs=false,
  %% stepnumber=5,
  tabsize=2,
  title=\lstname
}

%%%%%%%%%%%%%%%%%%%%%%%%%%%%% For Typesetting Coq %%%%%%%%%%%%%%%%%%%%%%%%%%%%%%

\definecolor{lightgray}{rgb}{.9,.9,.9}
\definecolor{darkgray}{rgb}{.4,.4,.4}

%% A minimal definition of the Coq language
\lstdefinestyle{Coq}{
  %% Gallina Keywords
  keywords = {
    _,
    as,
    at,
    cofix,
    else,
    end,
    exists,
    exists2,
    fix,
    for,
    forall,
    fun,
    if,
    IF,
    in,
    let,
    match,
    mod,
    Prop,
    return,
    Set,
    then,
    Type,
    using,
    where,
    with
  }
  sensitive    = true,
  keywordstyle = \color{black}\bfseries,
  %% Keywords for the Vernacular
  ndkeywords   = {
    Axiom,
    Conjecture,
    Parameter,
    Parameters,
    Variable,
    Variables,
    Hypothesis,
    Hypotheses,
    Definition,
    Let,
    Inductive,
    CoInductive,
    Fixpoint,
    CoFixpoint,
    Theorem,
    Lemma,
    Remark,
    Fact,
    Corollary,
    Program,
    Proposition,
    Example,
    Proof,
    Qed,
    Admit,
    Admitted
  }
  ndkeywordstyle  = \color{black}{\bfseries},
  identifierstyle = \color{black},
  morecomment     = [s]{(*}{*)},
  commentstyle    = \color{red}\ttfamily,
  stringstyle     = \color{darkgray}\ttfamily,
  morestring      = [b]',
  morestring      = [b]"
}

%% \lstnewenvironment{coq}{}{}

\newcommand{\secref}[1]{Section~\ref{sec:#1}}
\newcommand{\appref}[1]{Appendix~\ref{app:#1}}
\newcommand{\figref}[1]{Figure~\ref{fig:#1}}
\newcommand{\fmtlang}[1]{\mathbf{#1}}
\newcommand{\srclang}{\fmtlang{S}}
%% \newcommand{\intlang}{\fmtlang{B}}
\newcommand{\tgtlang}{\fmtlang{T}}

\newcommand{\fmtprog}[1]{\mathbf{#1}}
\newcommand{\srcprog}{\fmtprog{s}}
%% \newcommand{\intprog}{\fmtprog{b}}
\newcommand{\tgtprog}{\fmtprog{t}}
\newcommand{\compil}{\mathbb{C}}

\newcommand{\intel}{{\tt ia-32}}
\newcommand{\becomes}{\hspace{1cm}{ becomes ... }\hspace{1cm}}
\newcommand{\stepsto}[3]{\raisebox{-1ex}{\begin{tikzpicture}\node (0) {$#1$}; \node(1)[right of=0,xshift=1cm]{$#2$}; \draw[->] (0) -- node[above]{$#3$} (1);\end{tikzpicture}}}
\newcommand{\stepstoplus}[3]{\raisebox{-1ex}{\begin{tikzpicture}\node (0) {$#1$}; \node(1)[right of=0,xshift=1cm]{${}^+#2$}; \node(2)[left of=1,xshift=3mm]{$$}; \draw[->] (0) -- node[above]{$#3$} (1);\end{tikzpicture}}}

\newcommand{\bisim}{R}
\newcommand{\fwdsim}{{\bisim_\rightarrow}}
\newcommand{\backsim}{{\bisim_\leftarrow}}
\newcommand{\valid}{\mathbb{V}}
