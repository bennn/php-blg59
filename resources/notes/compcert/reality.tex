\section{Introduction}

Debugging is always a bummer.
This is especially true if buggy software starts losing money, releasing secrets, or killing people.
Hence the push for verified software: let's get it right the first time.

Embedded systems is a large application area where mistakes can be fatal.
Airplanes, automobiles, construction equipment, spaceships, and the like all run embedded systems.
Embedded systems have another constraint: the programs must follow a strict resource limit.
For these reasons, the C programming language is commonly used in embedded systems because of its proximity to the hardware and portability.

Debugging embedded systems code is very difficult.
The CompCert C compiler offers an efficient, optimizing, and safe C compiler~\cite{l-verified}.
CompCert claims that if it compiles a program, the executable will only perform only a subset of the actions one would expect the source code to perform.
This guarantee does not eliminate debugging\textemdash nor does it specify execution time or memory consumption\footnote{In practice, CompCert seems reasonably time and space efficient~\cite{website,airbus-slides}}\textemdash but it at least rules out translation bugs, thereby encouraging source-program reasoning.

High-assurance software is an old idea.
The UK-based Motor Industry Software Reliability Association maintains a specification for safe C programming~\cite{misra-spec}.
Languages like Rust (2009), Go (2007), and Ada (1977) are alternative low-level languages with stronger correctness guaranatees.
Companies like Wind River\footnote{http://www.windriver.com/} sell high-assurance tools, including the optimizing DIAB compiler toolchain for C.
Airbus has been operating \emph{fly-by-wire} planes since the 1987 release of the A320 line; more recently in Boeing released the fly-by-wire 777 in 1994 and the 787 in 2009.
These aircraft and their military predecessors were certainly running ``verified'' software\textemdash most likely manually-validated assembly~\cite{}.

Within the design space of high-assurance code, CompCert is the first \emph{formally verified} C compiler, in the sense that much of the compiler is written and proven correct in the Coq proof assistant.\footnote{The appendix in \secref{coq} gives a very short introduction to Coq.}
Whether to trust Coq more than a carefully-developed C program is a personal decision (see \secref{guarantees}), but the CompCert compiler is arguably more optimizing, extensible, and maintainable than previous verified compilers~\cite{airbus-slides}.


\subsection{Source Languages}

CompCert supports most of the {\bf ANSI C} and {\bf ISO-C-99} standards.
{\bf ANSI C} refers to the American standard published in 1990~\cite{ansi-c}.
{\bf ISO-C-99} refers to the international standard, published in 1999~\cite{iso-c99}.
(Suffice to say, these are widely used programming languages.)
We will make reference to {\bf ISO-C-99} as C99 throughout.

\subsubsection{Details and Caveats}
CompCert does not accept the entire C99 standard.
These are the most interesting exceptions and design details, as documented in Chapter 5 of the CompCert reference manual~\cite{refman}.

%% TODO double-check tense and suchthings
\begin{itemize}
\item
  CompCert follows the \emph{hosted environment} model, using the standard library of the target system.
  C99 defines a hosted implementation as one that accepts any strictly-conforming program~\cite{iso-c99}.
  In contrast, a \emph{free-standing} implementation can reject programs that use standard headers other than
   {\tt float.h, iso646.h, limits.h, stdarg.h, stdbool.h, stddef.h,}~and~{\tt stdint.h}.
\item
  Programs must be sequential.
  No {\tt pthreads}!
\item
  Preprocessing is done externally, by the system C compiler.
\item
  Multibyte characters are not allowed in source code; for example $\lambda$.
  %% This is technically a break from the standard.
  %% C99 states that the first translation stage of a source program should map multibyte characters to the source character set in an implementation-defined manner~\cite{iso-c99}.
\item
  The {\tt long double} type\footnote{CompCert can accept {\tt long double} as a synonym for {\tt double}. This compiles with C99, but may conflict with other binaries on some platforms.} and functions returning struct or union types are not allowed by default.
\item
  Complex types (like {\tt double \_Complex}) and variable-length arrays are not supported.
\item
  The {\tt restrict} keyword is ignored.
  Normally, this keyword marks a variable as the exclusive accessor for its underlying object.
\item
  The {\tt switch} statement is the same from Java and Misra-C~\cite{java-spec,misra-spec}.
  In particular, Duff's Device is not allowed (see \appref{A} for a definition).
\item
  The {\tt asm} statement (for inline assembly) may be enabled with a flag, and may be used to invalidate correctness guarantees.
\item
  The {\tt \_Alignof} and {\tt \_Alignas} operators from {\bf ISO-C-2011} are supported.
\item
  CompCert assumes that {\tt setjmp} and {\tt longjmp} respect control flow.
  All local variables that are live across an invocation of {\tt setjmp} must be declared {\tt volatile} to avoid mis-optimization~\cite{refman}.
\item
  The bit-level representation of integers and floats is exposed, but pointers are opaque 32-bit words.
\item
  Integers follow the ILP32LL64~\cite{ilp32ll64} model and floats follow the IEEE-754-2008~\cite{ieee-754} standard.
  To paraphrase, arithmetic must be done with infinite precision and finally rounded to a machine-dependent precision when stored to a variable.
\item
  The compiler will optimize floating point operations assuming that all floats are rounded to the nearest representable number, with ties broken by rounding to an even mantissa.
  Use a flag to disable this optimization if the rounding technique will change at runtime.
  %% ({\tt volatile} declares a variable as mutable)
\item
  The compiler will \emph{not} implicitly reorder floating-point computations.
  Operators like fused multiply-add are available as primitives and must be called explicitly.
\end{itemize}

Programs written in C are preprocessed, parsed, type-checked, and elaborated into the Clight language, for which CompCert has formal semantics.


\subsection{Target Languages}

CompCert provides 3 code generators~\cite{refman}:

\begin{itemize}
\item PowerPC 32 bits (all models)
\item ARM v6 and above with VFP coprocessor (for floating-point arithmetic)
\item Intel/AMD x86 in 32-bit mode, with SSE2 extension (for integer-vector operations)
\end{itemize}

PowerPC is supported on all 32-bit Linux distributions.
ARM is supported on Debian \& Ubuntu Linux (the {\tt armel} and {\tt armhf} architectures.
x86 works on all 32-bit and all 64-bit-emulating-32-bit Linux distributions, along with MacOS $\ge\!10.6$ and Windows via Cygwin.
The specific application binary interfaces and related limitations are documented in the reference manual~\cite{refman}.


\subsection{Compiler Structure}

CompCert is implemented in 25 passes~\cite{refman}.
The first 3 are pre-processing: these convert C source code to CompCert C syntax.
The next 19 are the compiler back-end, including 8 optimizing passes.
%% POPL'06 :
%%  (~6 passes) cminor -> internal cminor -> rtl -> ltl -> linear -> mach -> powerpc 
%% FM'06 :
%%  (~2 passes) clight -> cminor
The final 3 convert ASM syntax trees to raw assembly code, assemble an object file, and link to create an executable.

Of these passes, the parsing pass and 15 back-end passes are implemented and verified in Coq~\cite{refman}.
These constitute 90\% of the compiler.
File inclusion, macro expansion, conditional compilation, and other pre-processing steps are handled by an external C preprocessor.
Assembling and linking are also handled externally; however, PowerPC users can use the {\tt checklink} tool to assert basic properties about the linked code (i.e., linking did not remove any code blocks).


\subsubsection{Optimizing Passes}
These are the main optimizations, listed in the order that they appear as compiler passes.
See \secref{passes} for details.

\begin{multicols}{2}
\begin{enumerate}
\item Instruction selection
\item Tail call elimination
\item Function inlininng
\item Constant propogation
\item Common subexpression elimination
\item Dead code elimination
\item Branch tunneling
\item Register allocation
\end{enumerate}
\end{multicols}

The compiler does not perform any loop optimizations~\cite{refman}.
No hoisting, no reordering, no unrolling.


\subsection{Guarantees}
\label{sec:guarantees}

The compiler correctness theorem states, informally,\footnote{\secref{terminology} introduces necessary terminology and \secref{correctness} formally states the correctness theorem.}
 that compiled code \emph{refines} the \emph{behavior} of source code~\cite{refman}.
By \emph{refines}, we mean that all compiled behaviors are observable source behaviors with the exception that compiled code may eliminate dead code that caused the source code to diverge.
By \emph{behavior}, we mean the trace of all I/O and volatile (read: mutable) operations plus the program's exit code, if any.

The behavior of source and compiled code is specified in terms of the languages' formal semantics.
As C99 does not have a formal semantics, much less one implemented in Coq, we cannot even state a whole-compiler correctness theorem.
Current research aims to fix this issue~\cite{k-standard,klw-formal}.


\subsubsection{Why should we trust Coq?}

CompCert's correctness theorem is stated and proved in Coq.
If Coq is incorrect, then the theorem does not hold.
However:
\begin{itemize}
\item The Coq developers believe in Coq
\item The CompCert developers believe in Coq
\item The Airbus team believes in CompCert
\item John Regehr hasn't been able to break CompCert
\end{itemize}

% So we have some reason to believe that the Coq is correct, just as we can believe that the sun will rise tomorrow.

Speaking of John Regehr, his group famously wrote in their PLDI'11 paper~\cite{ycer-finding}:
\begin{quote}
  ``The striking thing about our CompCert results is that the middle end bugs we found in all other compilers are absent.
  As of early 2011, the under-development version of CompCert is the only compiler we have tested for which Csmith cannot find wrong-code errors.
  This is not for lack of trying: we have devoted about six CPU-years to the task.
  The apparent unbreakability of CompCert supports a strong argument that developing compiler optimizations within a proof framework, where safety checks are explicit and machine-checked, has tangible benefits for compiler users.''
\end{quote}

More recent bug-finding tools~\cite{las-compiler} have been unable to break the verified portions of CompCert as well.
Even if we do not fully trust Coq, we should at least prefer it over hand-validated C. 


\subsection{Installing CompCert}

CompCert is not free software, but a \href{https://github.com/AbsInt/CompCert}{non-commercial version} is on Github for ``educational, research or evaluation purposes only''.
After cloning the repository:
\begin{enumerate}
\item {\tt ./configure ia32-linux} \\
      (MacOS and Cygwin users: change to {\tt ia32-macosx} or {\tt ia32-cygwin})
\item {\tt make}
\item {\tt make install}
\end{enumerate}

You can then compile a program using the {\tt ccomp} executable.
The interface to {\tt ccomp} is nearly the same as {\tt gcc}, only restricted to the options that CompCert implements.

