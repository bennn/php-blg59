\documentclass{article}
\usepackage{hyperref}
\usepackage{palatino}
\usepackage{amsmath}
\usepackage{amsthm}
\usepackage{setspace}

\newtheorem*{theorem}{Theorem}

\newcommand{\summary}[1]{
\renewcommand{\maketitle}{{%
\flushleft\textsf{Ben Greenman \hfill \today} \\
\textsf{\hfill #1} \\
\vspace{1mm}
\hrule
\vspace{0.4cm}}}
\maketitle
}

\newcommand{\swtable}[2]{
\vspace{0.5cm}
\subsubsection*{Strengths}
\begin{itemize}
#1
\end{itemize}
\vspace{0.1cm}

\vspace{0.5cm}
\subsubsection*{Weaknesses}
\begin{itemize}
#2
\end{itemize}
\vspace{0.1cm}
}

\newcommand{\questions}[1]{
\vspace{0.5cm}
\subsubsection*{Questions}
\begin{itemize}
#1
\end{itemize}
\vspace{0.1cm}
}
\newcommand{\comments}[1]{
\vspace{0.5cm}
\subsubsection*{Comments}
\begin{itemize}
#1
\end{itemize}
\vspace{0.1cm}
}


\begin{document}
\maketitle

\subsubsection*{0. Quick Paper Summary}
\begin{itemize}
\item McMillan's definition
\item State the applications
\item[]
  \begin{itemize}
  \item Compute program-path invariants
  \item Choose predicates, for predicate abstraction
  \item Shrink a transition relation
  \end{itemize}
\end{itemize}


\subsubsection*{1. Background}
\begin{itemize}
\item Craig's definition \& lemma \textemdash ``The last deep theorem of first-order logic''
\item Examples, i.e. $P \wedge (P \Rightarrow Q)$ and $R \Rightarrow Q$ have interpolant $Q$.
\item Simple proof, by induction on $\big|\,\fvs{A} \setminus \fvs{B}\,\big|$
\item The ``art'' of picking a best interpolant
\end{itemize}


\subsubsection*{2. Programming Example}
\begin{itemize}
\item Key application: finding bad paths in a program
\item Correctness: all paths $A$ that reach assertion $B$ should imply $\vdash A \wedge B$.
\subitem i.e., we want $\vdash \neg(A \wedge \neg B)$
\item Interpolant for $(A, \neg B)$ should give ``minimal facts'' about $A$
      used to prove $\neg \neg B$. Use these ``facts'' to get inductive invariants.
\end{itemize}


\subsubsection*{3. Back to McMillan}
\begin{itemize}
\item Technical details of bounded model-checking
\item Inductive invariants, predicate abstraction, simpler transition relations
\item Infinite-state systems, Lamport's bakery \& Fischer's timed mutex
\item Limitation: non-local properties
\end{itemize}


\subsubsection*{4. Interpolants from Refutations}
\begin{itemize}
\item How to go from $\vdash \neg(A \wedge B)$ to an interpolant?
\item Which theories have efficient procedures, what is ``efficient''
\item Composing theories
\end{itemize}

\end{document}
