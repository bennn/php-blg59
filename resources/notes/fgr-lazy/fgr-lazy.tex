\documentclass{article}

\usepackage{hyperref}
\usepackage{listings}
\usepackage{palatino}
\usepackage{amsmath}
\usepackage{amsthm}
\usepackage{mathpartir}
\usepackage{graphicx}
\usepackage{setspace}
\usepackage{stmaryrd}

\newtheorem*{theorem}{Theorem}
\newtheorem*{challenge}{Challenge}
\newtheorem*{defn}{Definition}

\renewcommand{\maketitle}{{%
\flushleft\textsf{Ben Greenman}
\textsf{\hfill 2016-02-29 \hfill Outline: Interpolants in Model Checking} \\
\vspace{1mm}
\hrule
\vspace{0.4cm}}}

\newcommand{\fvs}[1]{\emph{atoms}(#1)}

\lstset{
  basicstyle=\ttfamily,
  breaklines=true,
  language=C,
  numbers=left
}

\newcommand{\todo}[1]{\textbf{TODO: #1}}


\begin{document}
\summary{Lazy Contract Checking for Immutable Data Structures}

The paper notes 2 desirable alternatives for data structure contracts:
\begin{enumerate}
\item Structures following an invariant should be subtypes of structures possibly-not following the invariant.
\item Contracts should not change the asymptotic complexity of an algorithm.
\end{enumerate}

The point of $1$ is data structures should not always be black-box.
If a library designer exposes his type representation, users should be able to synthesize data structures without using the library's constructors.
Also, it should be possible to re-use third-party functions on compatible data, even if the third-party does not necessarily preserve the data structure's invariants.

Supporting $1$ previously violated $2$, as every call to a library function would exhaustively check a data structure.
This paper solves the issue with lazy recursive contracts~\cite{fgr-lazy} 

The new operator is {\tt define-contract-struct}\footnote{As of Jan. 2016, Racket integrates dependent struct contracts with the conventional {\tt struct} keyword.}
Recursive structs are validated as they are explored\textemdash only when a field is accessed.

\emph{Caveat:} this works only for structs; the built-in {\tt vector} and {\tt hash} types, etc, are still eager.

\footnotesize
\bibliographystyle{plain}
\bibliography{fgr-lazy}
\end{document}
